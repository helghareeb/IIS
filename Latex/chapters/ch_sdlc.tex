\chapter{Systems Development Life Cycle (SDLC)}

\section{Introduction}
The \textbf{Systems Development Life Cycle (SDLC)} is a structured process used for developing and maintaining information systems. It provides a systematic approach to managing IT projects, ensuring efficiency, reliability, and effectiveness.

\section{Phases of the SDLC}
The SDLC consists of multiple phases, each with specific objectives:

\subsection{1. Initial Strategy}
\begin{itemize}
    \item Understand the problem that the system aims to solve.
    \item Investigate the environment of the company, project, and industry.
\end{itemize}

\subsection{2. Feasibility Study}
Determines whether the project is viable and worth pursuing based on different feasibility factors:
\begin{itemize}
    \item \textbf{Operational Feasibility}: Ensures user acceptance and system usability.
    \item \textbf{Technical Feasibility}: Assesses the capability of hardware and software.
    \item \textbf{Economic Feasibility}: Evaluates cost-benefit analysis.
    \item \textbf{Schedule Feasibility}: Determines whether the system can be completed within the required timeframe.
    \item \textbf{Organizational Feasibility}: Ensures alignment with business objectives.
    \item \textbf{Political Feasibility}: Analyzes management support.
    \item \textbf{Legal/Contractual Feasibility}: Ensures compliance with legal regulations.
\end{itemize}

\subsection{3. Requirements Analysis}
Identifies what the system should do by gathering requirements:
\begin{itemize}
    \item \textbf{Basic Functional Requirements}: Core system functions.
    \item \textbf{User Transaction Requirements}: User interactions with the system.
    \item \textbf{Decision-Making Requirements}: Analytical capabilities.
    \item \textbf{Organization-wide Requirements}: Integration across business units.
\end{itemize}

\subsection{4. Systems Analysis}
Examines the current system and identifies improvements. Techniques include:
\begin{itemize}
    \item Interviews
    \item Questionnaires
    \item Observations
    \item Record reviews
\end{itemize}

\subsection{5. Systems Specification}
A formal statement of what the system \textbf{will} do, ensuring clarity for users and developers. Characteristics include:
\begin{itemize}
    \item \textbf{Top-down approach}
    \item \textbf{Graphical representation}
    \item \textbf{Logical and precise descriptions}
\end{itemize}

\subsection{6. System Design}
Focuses on \textbf{how} the system will function. Key components:
\begin{itemize}
    \item \textbf{Logical Design}: Defines system functionality.
    \item \textbf{Physical Design}: Specifies the technical infrastructure.
    \item \textbf{Security Considerations}: Implements access controls and encryption.
\end{itemize}

\subsection{7. System Development}
\begin{itemize}
    \item System construction, programming, and testing.
    \item Decision between in-house development or purchasing commercial software.
\end{itemize}

\subsection{8. Testing}
Ensures the system meets business and technical requirements:
\begin{itemize}
    \item \textbf{Unit Testing}: Tests individual components.
    \item \textbf{System Testing}: Ensures integrated components work together.
    \item \textbf{Acceptance Testing}: Conducted by end-users before deployment.
\end{itemize}

\subsection{9. Implementation}
Deploying the system into production:
\begin{itemize}
    \item \textbf{Direct Implementation}: Immediate switch from old to new system.
    \item \textbf{Parallel Implementation}: Running both systems simultaneously.
    \item \textbf{Pilot Implementation}: Deploying in one department before company-wide rollout.
    \item \textbf{Phased Implementation}: Gradual rollout in stages.
\end{itemize}

\subsection{10. Production and Maintenance}
Ensures continuous system functionality:
\begin{itemize}
    \item \textbf{Emergency Maintenance}: Fixing critical issues.
    \item \textbf{Enhancement Maintenance}: Adding new features.
    \item \textbf{Environmental Maintenance}: Adapting to changing conditions.
\end{itemize}

\subsection{11. System Review}
Evaluates system performance and recommends improvements. Types of reviews:
\begin{itemize}
    \item \textbf{Project Review}: Assesses project execution.
    \item \textbf{System Review}: Evaluates user experience.
    \item \textbf{Periodic Review}: Ongoing assessments to ensure system relevance.
\end{itemize}

\section{Software Development Methodologies}
\begin{itemize}
    \item \textbf{Waterfall Model}: Sequential phase-by-phase development.
    \item \textbf{Rapid Application Development (RAD)}: Incorporates prototyping.
    \item \textbf{Extreme Programming (XP)}: Focuses on iterative, small reusable modules.
    \item \textbf{Agile Development}: Emphasizes fast, continuous delivery and customer collaboration.
\end{itemize}

\section{AI in SDLC}
AI enhances SDLC by:
\begin{itemize}
    \item Automating code generation.
    \item Enhancing testing through AI-driven debugging tools.
    \item Improving project estimations using machine learning.
\end{itemize}

\section{Case Study: AI-Assisted Software Development}
A global enterprise adopted AI-based software development tools to streamline requirement gathering, code analysis, and bug detection, reducing development time by 30\%.

\section{Multiple Choice Questions (MCQs)}
\begin{enumerate}
    \item What is the purpose of SDLC?
    \begin{enumerate}
        \item[A.] Manage business transactions
        \item[B.] Guide the development of information systems
        \item[C.] Control marketing strategies
        \item[D.] Monitor financial records
    \end{enumerate}
    \textbf{Answer: B}

    \item Which phase assesses project feasibility?
    \begin{enumerate}
        \item[A.] System Design
        \item[B.] Feasibility Study
        \item[C.] Testing
        \item[D.] Implementation
    \end{enumerate}
    \textbf{Answer: B}

    \item What is the primary goal of the Requirements Analysis phase?
    \begin{enumerate}
        \item[A.] Define what the system should do
        \item[B.] Design the software interface
        \item[C.] Perform system testing
        \item[D.] Execute project review
    \end{enumerate}
    \textbf{Answer: A}

    \item What is a characteristic of Agile methodology?
    \begin{enumerate}
        \item[A.] Sequential development
        \item[B.] Fixed design
        \item[C.] Iterative and customer-driven
        \item[D.] Limited project scope
    \end{enumerate}
    \textbf{Answer: C}

    \item What is the purpose of system testing?
    \begin{enumerate}
        \item[A.] To debug a single function
        \item[B.] To ensure the entire system functions properly
        \item[C.] To replace the project manager
        \item[D.] To perform financial forecasting
    \end{enumerate}
    \textbf{Answer: B}

    \item Which phase involves user acceptance testing?
    \begin{enumerate}
        \item[A.] Analysis
        \item[B.] Implementation
        \item[C.] Maintenance
        \item[D.] Testing
    \end{enumerate}
    \textbf{Answer: D}

    \item What does AI improve in SDLC?
    \begin{enumerate}
        \item[A.] System security only
        \item[B.] Automating code generation and testing
        \item[C.] Reducing employee workloads
        \item[D.] Limiting innovation
    \end{enumerate}
    \textbf{Answer: B}
\end{enumerate}
