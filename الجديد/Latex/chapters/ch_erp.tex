\chapter{Transaction Processing Systems (TPS) and Enterprise Resource Planning (ERP)}

\section{Introduction}
In modern organizations, information systems help manage business operations efficiently. Two essential types of systems in business environments are:
\begin{itemize}
    \item \textbf{Transaction Processing Systems (TPS)}: Handle routine business transactions.
    \item \textbf{Enterprise Resource Planning (ERP)}: Integrate business processes across multiple departments.
\end{itemize}

\section{Transaction Processing Systems (TPS)}
A \textbf{Transaction Processing System (TPS)} is an information system that collects, processes, stores, and retrieves transactions in an organization. Transactions include orders, payments, receipts, invoices, and payroll processing.

\subsection{Types of TPS}
There are two primary types of TPS:
\begin{itemize}
    \item \textbf{Batch Processing Systems}: Accumulate transactions over a period and process them in bulk.
    \item \textbf{Online Transaction Processing (OLTP)}: Processes transactions instantly, ensuring real-time updates.
\end{itemize}

\subsection{Comparison: Batch Processing vs. OLTP}
\begin{itemize}
    \item \textbf{Batch Processing}: Suitable for non-urgent, high-volume transactions such as payroll processing.
    \item \textbf{OLTP}: Best for real-time, mission-critical applications like online banking.
\end{itemize}

\section{Enterprise Resource Planning (ERP)}
An \textbf{Enterprise Resource Planning (ERP)} system is a collection of integrated, cross-functional systems that manage business processes across departments.

\subsection{Key Features of ERP Systems}
\begin{itemize}
    \item Centralized database for enterprise-wide data consistency.
    \item Integration of core business functions such as finance, HR, supply chain, and manufacturing.
    \item Real-time data access and reporting.
    \item Automation of business processes.
\end{itemize}

\section{Control and Management Issues in TPS and ERP}
Managing TPS and ERP systems requires careful planning and auditing.

\subsection{Business Continuity Planning (BCP)}
\begin{itemize}
    \item Identifies priorities for restoring business operations after disruptions.
    \item Specifies actions to restore critical systems.
\end{itemize}

\subsection{Transaction Processing System Audits}
\begin{itemize}
    \item Ensures data accuracy and system integrity.
    \item Uses an \textbf{Audit Trail} to track changes and trace outputs back to their source.
\end{itemize}

\section{International Issues in TPS and ERP}
\begin{itemize}
    \item \textbf{Language and Cultural Differences}: ERP systems must support multiple languages.
    \item \textbf{Information System Infrastructure}: Differences in IT capabilities across countries.
    \item \textbf{Telecommunication Quality}: Varies by region, affecting system performance.
    \item \textbf{Legal and Regulatory Compliance}: Different countries have different data privacy laws.
    \item \textbf{Multiple Currencies}: ERP systems must handle currency conversion.
\end{itemize}

\section{Case Study: AI-Driven ERP Systems}
AI-powered ERP systems enhance decision-making by analyzing historical data and predicting trends. Companies use AI-driven insights for inventory optimization, demand forecasting, and fraud detection.

\section{Multiple Choice Questions (MCQs)}
\begin{enumerate}
    \item What does TPS stand for?
    \begin{enumerate}
        \item[A.] Transaction Processing Software
        \item[B.] Transaction Processing System
        \item[C.] Total Process Solution
        \item[D.] Technical Processing Structure
    \end{enumerate}
    \textbf{Answer: B}

    \item Which type of TPS processes transactions in real-time?
    \begin{enumerate}
        \item[A.] Batch Processing
        \item[B.] OLTP
        \item[C.] Legacy Systems
        \item[D.] ERP
    \end{enumerate}
    \textbf{Answer: B}

    \item What is the main benefit of an ERP system?
    \begin{enumerate}
        \item[A.] Increases data redundancy
        \item[B.] Centralizes business data
        \item[C.] Creates isolated databases
        \item[D.] Only manages financial data
    \end{enumerate}
    \textbf{Answer: B}

    \item Which of the following is an example of batch processing?
    \begin{enumerate}
        \item[A.] Real-time stock trading
        \item[B.] Credit card authorization
        \item[C.] Payroll processing
        \item[D.] Online hotel booking
    \end{enumerate}
    \textbf{Answer: C}

    \item What does an audit trail help with?
    \begin{enumerate}
        \item[A.] Hiding unauthorized transactions
        \item[B.] Tracing system outputs back to their source
        \item[C.] Reducing system security
        \item[D.] Encrypting financial records
    \end{enumerate}
    \textbf{Answer: B}

    \item What is a key challenge of implementing ERP systems?
    \begin{enumerate}
        \item[A.] ERP systems require no maintenance
        \item[B.] They are inexpensive and easy to install
        \item[C.] Integration complexity and cost
        \item[D.] ERP systems do not support business automation
    \end{enumerate}
    \textbf{Answer: C}

    \item Which of the following is an international issue in ERP systems?
    \begin{enumerate}
        \item[A.] Multiple currencies and language barriers
        \item[B.] Lack of internet access in all countries
        \item[C.] No need for compliance with data privacy laws
        \item[D.] ERP systems only work in the United States
    \end{enumerate}
    \textbf{Answer: A}

    \item What is the role of AI in modern ERP systems?
    \begin{enumerate}
        \item[A.] Increases manual data entry
        \item[B.] Enhances predictive analytics and automation
        \item[C.] Slows down business processes
        \item[D.] Prevents cloud-based access
    \end{enumerate}
    \textbf{Answer: B}

    \item What is a major advantage of OLTP?
    \begin{enumerate}
        \item[A.] Processes transactions immediately
        \item[B.] Waits for transactions to accumulate before processing
        \item[C.] Does not require a database
        \item[D.] Works offline only
    \end{enumerate}
    \textbf{Answer: A}

    \item Which industry benefits most from ERP systems?
    \begin{enumerate}
        \item[A.] Manufacturing
        \item[B.] Banking
        \item[C.] Retail
        \item[D.] All of the above
    \end{enumerate}
    \textbf{Answer: D}
\end{enumerate}
